\documentclass[
	a4paper,
	11pt,
	oneside,
	openany,
	english,
	article
]{memoir}
\setFloatBlockFor{section} % enforce floats to not cross section borders
\usepackage{siunitx}
\usepackage{longtable}

\usepackage{multicol}
\usepackage[english]{babel}

\usepackage{listings}

\usepackage{lipsum}
\usepackage[iso]{isodate} % there is only one legit date format.
\usepackage[utf8]{inputenc}
\usepackage[T1]{fontenc} % umlaute im output als single chars

% damit die pdfreader page 1 bei mainmatter page 1 haben
\usepackage{zref-abspage,zref-user}
\makeatletter
\AtBeginDocument{%
  \zref@refused{pagenumstart}%
  \hypersetup{%
    pdfstartpage=\zref@extractdefault{pagenumstart}{abspage}{1}%
  }%
}
\makeatother

% suppress the page number after in the fronmatter/\maketitle
\usepackage{etoolbox}
\makeatletter
\aliaspagestyle{title}{empty}
\let\origps@chapter\ps@chapter
\preto\frontmatter{\let\ps@chapter\ps@empty\pagestyle{empty}}
\preto\mainmatter{%
  \cleardoublepage
  \let\ps@chapter\origps@chapter\pagestyle{headings}}
\makeatother

%\usepackage{aeguill} % schöne (vektor?)schriften in verbindung mit umlauten

\usepackage{graphicx}
\graphicspath{{./images/}}
\usepackage{epstopdf}
\epstopdfsetup{update,outdir=./output/images/}

% on the fly convert integrated svg to pdf
%http://wiki.inkscape.org/wiki/index.php/Tools#svg2pdf
% preamble: the svg
%\begin{filecontents*}{packets.svg}
%\input{images/packets.svg}
%\end{filecontents*}
% in the body something like:
%\begin{figure}
%\centering
%\immediate\write18{svg2pdf mypicture.svg  mypicture.pdf}%
%\includegraphics{mypicture.pdf}
%\caption{...}\label{...}
%\end{figure}

% attach files to pdf
%\usepackage{embedfile}

%\usepackage{array} % mehr table options
%\usepackage{ctable}
%\usepackage{threeparttable}
\usepackage{booktabs} % \toprule \midrule \bottomrule etc
\usepackage[table]{xcolor}
\definecolor{lightgray}{gray}{0.9}
\usepackage{multirow}
%\usepackage{rotating}
%\usepackage{pdflscape}
%\usepackage{diagbox}
%\usepackage{footnote}
%\usepackage{datatool}
%\renewcommand{\thempfootnote}{\arabic{mpfootnote}} % arabic footnotes in minipages

\usepackage{pgfplots}
\pgfplotsset{compat=1.5.1}
\usepackage{pgfplotstable}

\makeatletter
\newcommand\resetstackedplots{
\makeatletter
\pgfplots@stacked@isfirstplottrue
\makeatother
\addplot [forget plot,draw=none] coordinates{(1,0) (2,0) (3,0)} \closedcycle;
}
\makeatother

% source code listings
%\usepackage{listings}
%\usepackage{color} % colored fonts etc.
%\definecolor{listinggray}{gray}{1}
%\definecolor{colGrey}{rgb}{0.96,0.96,0.96}
%\definecolor{colKeys}{rgb}{0,0,0.5}
%\definecolor{colIdentifier}{rgb}{0,0,0}
%\definecolor{colComments}{rgb}{0,0.5,0}
%\definecolor{colString}{rgb}{0,0,1}
%\lstset{%
%language=XML,
%numbers=left,
%basicstyle=\fontfamily{pcr}\small,
%backgroundcolor=\color{colGrey},
%keywordstyle=\color{colKeys}\bfseries,
%identifierstyle=\color{colIdentifier},
%commentstyle=\color{colComments},
%stringstyle=\color{colString},
%columns=fixed,
%showstringspaces=true,
%}

%\usepackage{setspace}
%\usepackage{amsmath} % für \begin{align} für multiline equations
%\usepackage{amssymb} % für smileys und andere symbole
% \input kvmacros % für kv diagramme

\newenvironment{keywords}%
   {\begin{trivlist}\item[]{\bfseries\sffamily Keywords:}\ }
   {\end{trivlist}}

\newcommand{\tsub}{\textsubscript}
\newcommand{\tsup}{\textsuperscript}
\newcommand{\fixme}[1]{\textcolor{red}{[#1]}}

\bibliographystyle{alphaurl} % (custom) bibtex style

\copypagestyle{numberCorner}{plain}
\makeevenfoot{numberCorner}{\thepage}{}{}
\makeoddfoot{numberCorner}{}{}{\thepage}
\makeevenfoot{plain}{\thepage}{}{}
\makeoddfoot{plain}{}{}{\thepage}
\aliaspagestyle{chapter}{numberCorner}

\usepackage[stable,norule,multiple,marginal]{footmisc} % für footnotes in section headers, aber nicht in toc etc.
\makeatletter
\renewcommand\@makefntext[1]{\leftskip=3em\hskip-2em\@makefnmark~#1%
	\ifnum\the\spacefactor<3000.\fi} % add a full stop if there is none
\makeatother
%\usepackage{tablefootnote}
%\renewcommand\footnoterule{} % remove line above footnotes

\usepackage[pagebackref]{hyperref}
%\usepackage{bookmark}
\hypersetup{
	%pdftex,
	%pdfauthor={\tuinfthesisauthor},
	%pdftitle={\tuinfthesistitle},
	%pdfkeywords={\thesiskeywords},
	unicode=true,
	%pagebackref, % does not work, why?
	%pagebackref=true, % does not work, why?
	%hyperindex=true, % default
	%breaklinks=true, % default
	%plainpages=false, % default
	%colorlinks=true,	% false: boxed links; true: colored links
	%linkcolor=black,	% color of internal links
	%citecolor=black,	% color of links to bibliography
	%filecolor=black,	% color of file links
	%urlcolor=black,		% color of external links
	citebordercolor=0 0 0,	% hmpf!
	filebordercolor=0 0 0,
	linkbordercolor=0 0 0,
	menubordercolor=0 0 0,
	urlbordercolor=0 0 0,
	pdfborderstyle={/S/U /W 0.2}, % stroke, underline, width 0.2
	pdfstartview={FitV},
	%bookmarks=true, % default
	bookmarksnumbered=true,
	bookmarksopen=true,
	%hyperfootnotes=false,
}
\usepackage[all]{hypcap} % fix caption anchors
%\renewcommand{\backreflastsep}{, } % does not work (yet?)
%\renewcommand*{\backref}[1]{} % default interface disabled
%\renewcommand*{\backrefalt}[4]{%
	%1. Number of citations without dupes.
	%2. Back references list without dupes.
	%3. Number of all citations (with dupes).
	%4. Back reference list with all entries (with dupes).
	%Cited on page%
	%\ifnum#1>1 %
		%s%
	%\fi %
	%{:} #2{.}
%}

%\usepackage[
	%shortcuts,
	%toc,
	%translate=false,
	%acronym,
	%nostyles,
	%nomain
%]{glossaries}
%\usepackage{relsize} % needed for glossaries' smaller

%\newcommand*{\acro}[3]{%label, abbr, text
%\newacronym[user1={#3}]{#1}{#2}{#3}%
%}
%\newcommand*{\acrol}[4]{%label, abbr, text, desc
%\newacronym[user1={#3},description={#4}]{#1}{#2}{\glsentryuseri{#1}}%
%}
%\newcommand*{\acrop}[4]{%label, abbr, text, plural
%\newacronym[user1={#3},\glslongpluralkey={#4}]{#1}{#2}{\glsentryuseri{#1}}%
%}
%\newcommand*{\acrolp}[5]{%label, abbr, text, desc, plural
%\newacronym[user1={#3},description={#4},\glslongpluralkey={#5}]{#1}{#2}{\glsentryuseri{#1}}%
%}
%
% to get rid of the additional "Acronyms" title
%\makeatletter
%\renewcommand{\glossarysection}[2][\@gls@title]{%
%\def\@gls@title{#2}%
%\begin{multicols}{2}%
%\@ifundefined{phantomsection}{%
%\@glossarysection{#1}{#2}}{\@p@glossarysection{#1}{#2}}%
%}
%\makeatother
%\setlength{\columnsep}{1cm}
%\renewcommand*{\glossarypostamble}{\end{multicols}}
%\renewcommand*{\glsclearpage}{\clearpage} % instead of cleardoublepage

% customize the acronym printing to include the long form AND the description
%\newglossarystyle{mylist}{%
  %\renewenvironment{theglossary}%
    %{\begin{description}}{\end{description}}%
  %\renewcommand*{\glossaryentryfield}[5]{%
    %\item[\glstarget{##1}{##2}]\glsentryuseri{##1}\space {\footnotesize (pp. ##5)}%
      %\ifthenelse{\equal{\relax ##3}{\relax\glsentryuseri{##1}}}{}{\newline ##3\glspostdescription}%
  %}%
  %\renewcommand*{\glossaryheader}{}%
  %\renewcommand*{\glsgroupheading}[1]{}%
  %\renewcommand*{\glossarysubentryfield}[6]{%
    %\par\glstarget{##2}{\strut}##4\glspostdescription\space ##6%
  %}%
  %\renewcommand*{\glsgroupskip}{}%
%}
%\makeglossaries

%\newcommand{\X}[1]{#1\index{#1}} % shortcut for index
%\newcommand{\Xtt}[1]{\texttt{#1}\index{#1}} % shortcut for index+typewriter fonts... used to index classes and methods
%\makeindex
